% =======================================================================================
% =======================================================================================
% === IMPORTANT NOTE FOR STUDENTS:                                                    ===
% =======================================================================================
% === Do NOT change anything in this file. Only change files in the "reportContent"   ===
% === subfolder.                                                                      ===
% =======================================================================================
% =======================================================================================

\documentclass[runningheads,a4paper]{llncs}

\usepackage{makeidx}
\usepackage{chapterbib}
%\usepackage[english,german]{babel}
%\selectlanguage{german}

\usepackage[utf8]{inputenc}
\usepackage[T1]{fontenc}


%
%
% Note for the assistants: Add the different image paths of the student groups.
%
\usepackage{graphicx}\graphicspath{{images/}{reportContent/images/}%{reportContent\_g1/images/}{reportContent\_g2/images/}...
}


\usepackage{import}
\usepackage{lipsum}
\usepackage{setspace}
\usepackage{pdfpages}

% =======================================================================================
% === The following latex-packages should cover all your needs.                       ===
% === If this is not the case, please write an email to one of the assistants in      ===
% === which you inform us which additional package you need and why.                  ===
% === (Otherwise we will not be able to generate proceedings for the course...)       ===
% =======================================================================================


% For URLs
\usepackage{url}
\usepackage[hidelinks]{hyperref}

% For Subfigures
\usepackage{subfig}

% For Algorithms
\usepackage{algorithmic}
\usepackage{algorithm}

% For Code Snippets
\usepackage{listings}

% For Todo Notes 
\usepackage[colorinlistoftodos]{todonotes}

% For Colors (\textcolor etc.)
\usepackage{color}

% For Tables
\usepackage{multirow}

% For Tikz
\usepackage{tikz}

% For References
\usepackage{cleveref}

% Required for turning proceedings on and off
% (Source: http://tex.stackexchange.com/questions/87656/turning-parts-of-text-on-and-off)
\usepackage{etoolbox}
\usepackage{verbatim}\newbool{produceProceedings}
\newbool{produceProceedingsBookVersion}
\newenvironment{produceProceedings}{}{}

% For foreign key uwaves
\usepackage{ulem}
\normalem


%
%
% procedureProceedings : boolean
% ==============================
%
% If true: Generates the proceedings.
% If false: Generates the plain article (typically student mode).
%
% _______________________
% Note for the assistants: Change to true in order to generate the proceedings.
%
\setbool{produceProceedings}{false}


\ifbool{produceProceedings}{


%
%
% produceProceedingsBookVersion : boolean
% =======================================
%
% If true: Inserts empty pages so that every report start on an odd page which helps to print the proceedings as double pages.
% If false: No empy pages are inserted.
%
% _______________________
% Note for the assistants: Change to true in order to generate the proceedings which adds empty pages to start every report on an odd page.
%
\setbool{produceProceedingsBookVersion}{false}



}{\AtBeginEnvironment{produceProceedings}{\comment}\AtEndEnvironment{produceProceedings}{\endcomment}}


%
%
% coursename : String
% ===================
%
% Name of the course / lecture.
%
\newcommand{\coursename}{Databases}


%
%
% courseacronym : String
% ======================
%
% Acronym of the course (e.g., DIS for Distributed Information Systems, or CS244 for Databases).
%
\newcommand{\courseacronym}{CS244}


%
%
% semester : String
% =================
%
% The semester (e.g., spring semester 2018 or fall semester 2018).
%
\newcommand{\semester}{fall semester 2018}


%
%
% semesterCapitals : String
% =========================
%
% The semester with startring capital letters.
%
\newcommand{\semesterCapitals}{Fall Semester 2018}


%
%
% proceedingsSubtitle : String
% ============================
%
% Subtitle of the proceedings.
%
\newcommand{\proceedingsSubtitle}{Data Analysis Project Reports}


%
%
% rootDocument : String
% =====================
%
% Set the to this document's file name.
%
\newcommand{\rootDocument}{main.tex}

\begin{document}


\institute{University of Basel \\ \coursename\ (\courseacronym) course \\ \semesterCapitals} % Please do NOT change this line.

	
\begin{produceProceedings}
	
	\begin{titlepage}
		\includegraphics{UniBas_Logo_EN_Schwarz_RGB_65}
		
		\vspace{80pt}
		
		\centering
		
		\begin{spacing}{2.8}
		{\Huge \fontfamily{phv} \bfseries Proceedings of the \coursename\ (\courseacronym) Course}
		\end{spacing}
		
		\vspace{25pt}
		
		{\large \proceedingsSubtitle}
		
		\vspace{50pt}
		
		{\large \semesterCapitals}
		
		\vspace{50pt}
		
		{\large	Department of Mathematics and Computer Science}
		
		\vspace{5pt}
		
		{\large Faculty of Science}
		
		\vspace{25pt}
		
		{\large	University of Basel}
	\end{titlepage}
	
	\ifbool{produceProceedingsBookVersion}{\frontmatter}{}
	
	\pagestyle{headings}
	\addtocmark{Reports}
	%
	\chapter*{Preface}
	This documents contains all student reports of the \coursename\ course (\courseacronym) held at the University of Basel in the \semester.
	%
	\chapter*{Organization}
	The \coursename\ (\courseacronym) course of the \semester\ is organized by the Databases and Information Systems (DBIS) research group, Department of Mathematics and Computer Science, University of Basel.
	%
	\section*{Lecturer}
	Prof.\,Dr. Heiko Schuldt (heiko.schuldt@unibas.ch) \\
	Dr. Ivan Giangreco (ivan.giangreco@unibas.ch)
	%
	\section*{Assistants}
	Lukas Probst, M.\,Sc. (lukas.probst@unibas.ch) \\
	Alexander Stiemer, M.\,Sc. (alexander.stiemer@unibas.ch)
	%
	\section*{Tutors}
	Loris Sauter, B.\,Sc. (loris.sauter@stud.unibas.ch)
	%
	\tableofcontents
\end{produceProceedings}
	
%
\mainmatter
%

% Note for the assistants: Add all student reports to this list in order to generate the proceedings.
\ifbool{produceProceedingsBookVersion}{\cleardoublepage}{} \begin{cbunit}\import{reportContent/}{reportContent.tex}\end{cbunit}
%\ifbool{produceProceedingsBookVersion}{\cleardoublepage}{} \begin{cbunit}\import{reportContent\_g1/}{reportContent.tex}\end{cbunit}
%\ifbool{produceProceedingsBookVersion}{\cleardoublepage}{} \begin{cbunit}\import{reportContent\_g2/}{reportContent.tex}\end{cbunit}
%...

\end{document}

% =======================================================================================
% =======================================================================================
% === IMPORTANT NOTE FOR STUDENTS:                                                    ===
% =======================================================================================
% === Do NOT change anything in this file. Only change files in the "reportContent"   ===
% === subfolder.                                                                      ===
% =======================================================================================
% =======================================================================================